\documentclass[11pt]{article}
%\usepackage[firstpage]{draftwatermark}
\usepackage{times}
\usepackage{pdfpages}
\usepackage{fullpage}
\usepackage{url}
\usepackage{hyperref}
\usepackage{fancyhdr}
\usepackage{graphicx}
\usepackage{tabularx}
\usepackage{enumitem}
\usepackage{indentfirst}
\usepackage{subcaption}
\usepackage{units}

% Highlighting
\usepackage{color,soul}
\DeclareRobustCommand{\hlr}[1]{{\sethlcolor{red}\hl{#1}}}
\DeclareRobustCommand{\hlg}[1]{{\sethlcolor{green}\hl{#1}}}
\DeclareRobustCommand{\hlb}[1]{{\sethlcolor{blue}\hl{#1}}}
\DeclareRobustCommand{\hly}[1]{{\sethlcolor{yellow}\hl{#1}}}

\setcounter{secnumdepth}{4}
\graphicspath{{images/}}
\pagestyle{fancy}

% Conditional for notes
\newif\ifnotes
\notesfalse


\newcommand{\doctitle}{Wind Plugin Theory of Operation}

\newcommand{\docnumber}{\doctitle}

\addtolength{\headheight}{2em}
\addtolength{\headsep}{1.5em}
\lhead{\docnumber}
\rhead{}

\newcommand{\capt}[1]{\caption{\small \em #1}}

\cfoot{\small Brian Bingham \today \\ \thepage}
\renewcommand{\footrulewidth}{0.4pt}

\newenvironment{xitemize}{\begin{itemize}\addtolength{\itemsep}{-0.75em}}{\end{itemize}}
\newenvironment{tasklist}{\begin{enumerate}[label=\textbf{\thesubsubsection-\arabic*},ref=\thesubsubsection-\arabic*,leftmargin=*]}{\end{enumerate}}
\newcommand\todo[1]{{\bf TODO: #1}}
\setcounter{tocdepth}{2}
\setcounter{secnumdepth}{4}

\makeatletter
\newcommand*{\compress}{\@minipagetrue}
\makeatother

%\renewcommand{\chaptername}{Volume}
%\renewcommand{\thesection}{\Roman{section}}
%\renewcommand{\thesubsection}{\Roman{section}-\Alph{subsection}}

\begin{document}

\newpage
% Title Page
\setcounter{page}{1}
\begin{center}
{\huge \doctitle}
\end{center}

\section{Overview}

The influence of wind on the motion of the USV is implemented as a simple model plugin for Gazebo.  Currently there is only one implementation, but other implementations, with varying fidelity, could be developed in the future.


\section{usv gazebo wind plugin}

The wind forces (x and y) and moment (yaw) are predicted following the models presented by Fossen~\cite{fossen94guidance}.

The wind velocity on the vessel ($V_w$) is considered to be a constant velocity and direction.  If desired, this could be extended to include a parameterized wind spectrum the distribution of wind velocities over time, e.g., average wind velocity, gusts, etc.  For the current implementation the constant wind velocity is specified as a three element vector which specifies the wind speed the world-frame x, y and z coordinates with units of \unitfrac[]{m}{s}.  The z component is ignored.

The resulting forces and moments on the vessel are determined based on the user-specified force/moment coefficients and the relative wind velocity.  Within the plugin, the relative (or apparent) wind velocity vector $V_R$.  The forces/moment are calculated as
\begin{eqnarray}
  X_{wind} &=& C_X V_{R_x} |V_{R_x}| \\
  Y_{wind} &=& C_Y V_{R_y} |V_{R_y}| \\
  N_{wind} &=& -2.0 C_N V_{R_x} V_{R_y} \\
\end{eqnarray}
where $C_X$, $C_Y$ and $C_N$ are specified as the three element \verb+wind_coeff_vector+.  Approximate values for these coefficients are given in \cite{sarda17station} which can then be tuned to give reasonable response.

%\newpage
%\setcounter{page}{1}
\bibliographystyle{ieeetr}
\bibliography{refs}

\end{document}
